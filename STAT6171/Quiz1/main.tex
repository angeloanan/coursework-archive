\documentclass[
	11pt, % 10pt - 12pt
	%letterpaper
	%indonesian
]{assignment}

% Template-specific packages
\usepackage{mathpazo} % Use the Palatino font

\usepackage{graphicx} % Required for including images
\usepackage{booktabs} % Required for better horizontal rules in tables

\usepackage{amsmath} % Math!
\usepackage{listings} % Required for insertion of code
\usepackage{enumerate} % To modify the enumerate environment

\newcommand{\ipAddress}[1]{{\fontfamily{cmtt}\selectfont #1}} % IP Address custom style!

% ! CUSTOM - LST Preset
\lstset{
    language=SQL,
    frame=single, % Frames
    showstringspaces=false, % Don't put marks in string spaces
    numbers=left, % Line numbers on left
    numberstyle=\tiny, % Line numbers styling
    breaklines,
    basicstyle=\fontfamily{cmtt}\selectfont\small,
    columns=fullflexible,
}


%----------------------------------------------------------------------------------------
%	ASSIGNMENT INFORMATION
%----------------------------------------------------------------------------------------

% Student name
\author{Christopher Angelo - 2440041503}
% Institute or school name
\institute{BINUS University\\ Global Class}


% Due date
\date{April 7th, 2022}
% Assignment title
\title{Quiz 1 Answer}

% Course details
\class{Basic Statistics (STAT6171)}
\professor{Ms.\ Diah Wihardini}

%----------------------------------------------------------------------------------------

\begin{document}
\maketitle

%----------------------------------------------------------------------------------------
%	ASSIGNMENT CONTENT
%----------------------------------------------------------------------------------------

\section*{Question 1}

\begin{problem}
The University of Southern California conducted a survey to investigate whether people of different age groups differ in their response to e-mail messages (The New York Times, 2008). Based on a sample of 2000 email users, the table below lists the sample data based on the age of respondents and whether or not they answered quickly.
\end{problem}

\subsection*{Answer 1}

\begin{enumerate}
	\item We can also enumerate for ordered list
\end{enumerate}
\begin{itemize}
	\item Or if you want, you can use itemize for unordered lists
\end{itemize}

\begin{flalign*}
	\text{Here is some fancy aligned stuff } & \text{= \textbf{CHRIST}}         \\
	\text{Here is some fancy aligned stuff } & \text{= [C, H, R, I, S, T]}      \\
	\text{Here is some fancy aligned stuff } & \text{= [2, 7, 17, 11, 18, 19]}  \\
	\text{Here is some fancy aligned stuff } & \text{= [8, 13, 23, 17, 24, 25]} \\
	\text{Here is some fancy aligned stuff } & \text{= [i, n, x, r, y, y]}      \\
	\\
	\text{Decrypted message }                & \text{= \textbf{inxryz}}
\end{flalign*}

\lstinputlisting[]{code/.gitkeep}

% \printbibliography[]{}


\end{document}
